%---------- Inleiding ---------------------------------------------------------
\section{Inleiding}% 
\label{sec:inleiding}

Deze bachelorproef onderzoekt hoe een geautomatiseerd systeem voor security monitoring en incident response kan worden ontworpen om securitydata binnen SaaS-omgevingen effectief te centraliseren, analyseren en rapporteren in een DevOps-infrastructuur. 

Het onderzoek vertrekt vanuit een concrete casus bij Devinity.eu, een SaaS-bedrijf dat momenteel meerdere tools gebruikt voor security monitoring en incidentdetectie. Hierdoor ontstaan problemen zoals vertraagde detectie, onvolledige rapporten en inconsistentie in incidentafhandeling. De doelgroep bestaat uit DevOps- en securityteams binnen SaaS-bedrijven, die baat hebben bij een gecentraliseerd en geautomatiseerd systeem dat de operationele efficiëntie verhoogt en risico’s beter beheersbaar maakt.

De centrale onderzoeksvraag luidt: \textbf{Hoe kan een geautomatiseerd systeem voor security monitoring en incident response worden ontworpen om securitydata binnen SaaS-omgevingen effectief te centraliseren, analyseren en rapporteren in een DevOps-infrastructuur?}

De doelstelling van deze bachelorproef is het ontwikkelen van een proof-of-concept van een gecentraliseerd systeem dat:
\begin{itemize}
    \item Securitydata uit SaaS-applicaties (zoals Microsoft 365, Google Workspace, Okta) en DevOps-omgevingen verzamelt via API’s, webhooks en logforwarders;
    \item Data normaliseert en analyseert om incidenten sneller en accurater te detecteren;
    \item Automatische rapportages genereert voor technische teams en management;
    \item KPI’s levert om de effectiviteit van monitoring en incident response te meten.
\end{itemize}

Het concrete eindresultaat is een werkend prototype dat de voordelen van centralisatie, snelle detectie en consistente rapportage aantoont. 

%---------- Literatuurstudie ---------------------------------------------------
\section{Literatuurstudie}%
\label{sec:literatuurstudie}

Deze literatuurstudie behandelt de state-of-the-art van security monitoring, incident response en data-analyse binnen SaaS-omgevingen en DevOps. Het doel is een theoretische basis te leggen voor het ontwerp van het proof-of-concept en de selectie van technologieën. 

\subsection{Security monitoring \& SIEM}
Traditionele intrusion detection- en prevention-systemen (IDPS) vormen de basis van security monitoring. Volgens Scarfone \& Mell (2007) zijn detectie, correlatie van events en het beperken van false positives kernfuncties van IDPS-systemen \autocite{Scarfone2007}. Moderne SIEM-oplossingen zoals Elastic Security en OpenSearch Security combineren log ingestion, correlatie, dashboards en alerting in één platform. Deze systemen zijn goed toepasbaar binnen DevOps-omgevingen en vormen de basis voor geautomatiseerde monitoring.

\subsection{SaaS \& cloud logging}
SaaS-omgevingen bieden logging- en securitymonitoring via API’s, webhooks en agents. De Cloud Security Alliance benadrukt het belang van gecentraliseerde logging en governance \autocite{CSA2019}. Microsoft 365 en Okta documenteren welke logs en events beschikbaar zijn en hoe deze kunnen worden verzameld. Dit toont aan dat centralisatie van securitydata uit SaaS-diensten praktisch haalbaar is.

\subsection{Correlatie, threat intelligence \& detectie}
Het normaliseren van data en toevoegen van context is cruciaal voor effectieve detectie. Mavroeidis \& Bromander (2017) beschrijven taxonomieën en modellen voor cyber threat intelligence, wat helpt bij het structureren van alerts \autocite{Mavroeidis2017}. Het MITRE ATT\&CK Framework ondersteunt het definiëren van detectieregels en analyse van incidenten, waardoor een systematische aanpak mogelijk is.

\subsection{Metrics \& rapportering}
Het meten van prestaties en effectiviteit is essentieel. NIST SP 800-55 \autocite{NIST80055} en Jaquith (2007) beschrijven KPI’s zoals Mean Time To Detect (MTTD), Mean Time To Respond (MTTR) en aantal false positives. Deze metrics zijn bruikbaar voor zowel technische teams als management om de effectiviteit van het systeem te evalueren.

%---------- Methodologie ------------------------------------------------------
\section{Methodologie}%
\label{sec:methodologie}

Het onderzoek wordt uitgevoerd in vier fasen:

\begin{enumerate}
    \item \textbf{Probleemanalyse:} Samenwerking met Devinity.eu om huidige monitoringtools, workflows en tekortkomingen in kaart te brengen. Analyse van bestaande logs, incidenten en KPI’s.
    \item \textbf{Architectuurontwerp:} Selectie van technologieën zoals Elastic Stack, Wazuh en OpenSearch voor het verzamelen, normaliseren en analyseren van data. Definiëren van datastromen via API’s, webhooks en logforwarders.
    \item \textbf{Proof-of-concept ontwikkeling:} Implementatie van centrale logging, correlatie- en analysemethoden (rule-based en eenvoudige machine learning), en automatische rapportage. Testen binnen een DevOps-pipeline.
    \item \textbf{Evaluatie en KPI-meting:} Meten van detectiesnelheid, aantal false positives, responstijd en percentage automatisch afgehandelde incidenten. Analyse van resultaten en aanbevelingen voor optimalisatie.
\end{enumerate}

\textbf{Tijdsplanning:}
\begin{itemize}
    \item Fase 1: 3 weken – requirements-analyse en dataverzameling.
    \item Fase 2: 2 weken – architectuurontwerp en selectie van tools.
    \item Fase 3: 6 weken – implementatie van proof-of-concept.
    \item Fase 4: 2 weken – evaluatie, metingen en rapportering.
\end{itemize}

De deliverables per fase zijn:
\begin{itemize}
    \item Fase 1: Analyseverslag van huidige situatie en probleemstelling.
    \item Fase 2: Ontwerpspecificatie van de architectuur en datastromen.
    \item Fase 3: Werkend proof-of-concept systeem.
    \item Fase 4: Evaluatierapport met KPI’s, conclusies en aanbevelingen.
\end{itemize}

%---------- Verwachte resultaten ----------------------------------------------
\section{Verwacht resultaat, conclusie}%
\label{sec:verwachte_resultaten}

Het verwachte resultaat is een proof-of-concept van een gecentraliseerd securitymonitoringsysteem dat:

\begin{itemize}
    \item Securitydata uit meerdere SaaS- en DevOps-bronnen centraliseert en normaliseert;
    \item Incidenten sneller en betrouwbaarder detecteert via correlatie en analyse;
    \item Automatische rapportages genereert voor technische teams en management;
    \item KPI’s levert zoals MTTD, MTTR, aantal false positives en responstijd.
\end{itemize}

De meerwaarde voor de doelgroep (DevOps- en securityteams bij SaaS-bedrijven) ligt in hogere operationele efficiëntie, betere risico-inschatting en snellere incidentafhandeling. Het onderzoek levert zowel praktische inzichten als een concrete demonstratie van een geautomatiseerd monitoring- en incident response-systeem.
